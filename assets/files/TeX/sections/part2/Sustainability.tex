\section{Sustainability} \label{sec:sed-ultrices}


In recent years, we have witnessed an extraordinary acceleration in the growth of artificial intelligence (AI), transforming the way we live, work, and interact with technology. AI algorithms impact sectors such as healthcare, finance, manufacturing, transportation, and entertainment. These advances are driving new chip and server technologies, resulting in extreme rack power densities and presenting new challenges in the design and operation of data centers to meet the massive demand for AI. As part of their Environmental, Social, and Governance (ESG) programs, data center operators are making commitments to environmental sustainability.\\

Data center operators should use a standard set of metrics. The Green Grid (TGG) proposed power usage effectiveness (PUE) in 2007, which was widely adopted and helped drive efficiency improvements across the industry. A global survey conducted by Uptime Institute in 2023 showed that the average annual PUE of large data centers improved from 2.5 to 1.58 since 2007.\\

Schneider Electric's growth predictions estimate that AI accounts for 8\% of the total power consumption of data centers in 2023, rising to 15-20\% by 2028. Given that data centers worldwide consumed over 500 TWh of electricity in 2023 and are projected to consume 815 TWh in 2028 (the equivalent of 16 New York Cities), AI's power consumption equates to 40 TWh in 2023 and is expected to increase to between 122 and 163 TWh by 2028.


\section{Ground Truth Data Solution}

\subsection{Approach}

At Ground Truth Data, we're tackling energy consumption in data centers with our efficient data reduction technology. This approach not only cuts down on storage space for AI data but also reduces computing times and network loads, boosting overall system efficiency.\\

Our solution brings a range of benefits: better use of storage resources, cost savings, longer hardware lifespans, and optimized data processing. For AI applications, our techniques improve performance and scalability, allowing for faster training and deployment of AI models, especially deep learning.\\

Additionally, our technology speeds up backup and recovery processes, which is crucial for the finance and healthcare sectors, and enhances the efficiency of IoT devices in edge computing. For cloud service providers, it helps meet sustainability goals and lowers operational costs.\\

Using our data reduction strategies leads to significant energy savings, lower operational costs, and greater environmental sustainability, all while maintaining high performance and reliability. Our solution also supports AI development, making AI technologies more accessible and efficient across various industries.

\subsection{Competitors}
bla bla

\section{Results}

Ground Truth Data's solution achieves data reductions of up to 86\%, leaving only 14\% of the original data. This reduction translates to a 36\% savings in computing for AI algorithms on IoT devices and an 86\% savings in data centers.\\


Considering these figures and the optimistic consumption forecasts for data centers in the coming years, we can illustrate the potential energy and cost savings for the industry in the following table:

\begin{table}[H]
    \centering
    \footnotesize
    \begin{tabular}{|lrrrrrr}
        \toprule
        Year & AI (TWh) & Elect. price (\$) & AI Cost (B\$) & AI E. saved (TWh) & Cost saved (B\$)  & $CO_2$ saved (Mt)  \\
        \midrule
        2023 & 40  & 0.17  & 7  & 34  & 6  & 7  \\
        2024 & 54  & 0.175 & 9  & 46  & 8  & 10 \\
        2025 & 72  & 0.18  & 13 & 62  & 11 & 13 \\
        2026 & 92  & 0.185 & 17 & 79  & 15 & 16 \\
        2027 & 115 & 0.19  & 22 & 99  & 19 & 20 \\
        2028 & 143 & 0.195 & 28 & 123 & 24 & 25 \\
        2029 & 175 & 0.2   & 35 & 151 & 30 & 31 \\
        2030 & 212 & 0.206 & 44 & 182 & 38 & 38 \\
        \midrule
        & & & & 776 (TWh) & 151 (B\$) & 160 Mt $CO_2$ \\
        \bottomrule
    \end{tabular}
    \caption{}
    \label{tab:ai_summary}
\end{table}
