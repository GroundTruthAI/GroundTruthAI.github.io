%\section{Revenue}
The revenue table forecasts profits through 2026. The AI (TWh) field shows the estimated energy demand for AI processes in data centers worldwide, based on predictions from Schneider Electric. The Electricity price (\$) field predicts the average electricity price in the United States. 

The AI cost (B\$) field predicts the global cost of energy demand for AI processes in data centers. The AI energy saved (TWh) field estimates the energy savings from our solution. The Saved cost (B\$) field represents the predicted cost savings due to our solution, and the CO$_2$ saved (Mt) field shows the megatons of CO$_2$ emissions that will be prevented with our solution. SAM (\%) indicates the percentage of market share defining the serviceable addressable market, while SAM (B\$) represents the money saved in the SAM. SOM (M\$) predicts the serviceable obtainable market, which is the estimated profit until 2026 (comission 10\%). Cum. SOM (M\$) is the cumulative predicted profit expected until 2026. Following the upward trend in average electricity prices and the growing demand in data centers, it is estimated that by 2030, SOM benefits will reach at least 606 million dollars.

\begin{table}[!h]
\footnotesize
\begin{tabular}{ccccccccccccc}
\toprule
 & AI & Electricity & AI cost & AI energy  & Saved cost & CO\textsubscript{2}  & SAM & SAM  & SOM  & Cum. \\
Year & (TWh) & price (\$) & (B\$) & saved (TWh) &  (B\$) & saved (Mt) & (\%) &   (M\$) & (M\$) & SOM (M\$) \\
\midrule
2023 & 40   & 0.12   & 5   & 34  & 4   & 7   & 0.000     & 0     & 0    & 0 \\
2024 & 54  & 0.11  & 6  & 46    & 5  & 10  & 0.000     & 0      & 0    & 0 \\
2025 & 72 & 0.11   & 8  & 62    & 7  & 13  & 0.001 	& 7  & 1 & 1 \\
2026 & 92 & 0.11  & 10  & 79  & 9  & 16  & 0.007 	& 63   & 6 & 7 \\
2027 & 115 & 0.10  & 13  & 99  & 11  & 20  & 0.015 	& 165   & 17 & 24 \\
\bottomrule
\end{tabular}
\caption{Energy demand prevision and costs}
\end{table}

Given the increasing power demand in data centers and the anticipated scarcity of energy in the midterm future, our data size reduction solution should be considered a standard protocol. The adoption of our solution in the algorithmic AI processes of data centers leads to significant energy savings, which translates to substantial cost reductions and a decrease in CO$_2$ emissions. 

As energy demands continue to rise, the necessity for efficient energy solutions becomes more critical. Our solution offers a sustainable path forward, ensuring that data centers can meet the growing needs while minimizing their environmental impact and operational costs. The projected savings and reduced carbon footprint underscore the vital role our solution will play in the future of data center operations. Adopting our data size reduction as a standard protocol is essential for maintaining energy efficiency and sustainability in the face of increasing energy constraints.
